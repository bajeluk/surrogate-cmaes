
%% bare_conf.tex
%% V1.4a
%% 2014/09/17
%% by Michael Shell
%% See:
%% http://www.michaelshell.org/
%% for current contact information.
%%
%% This is a skeleton file demonstrating the use of IEEEtran.cls
%% (requires IEEEtran.cls version 1.8a or later) with an IEEE
%% conference paper.
%%
%% Support sites:
%% http://www.michaelshell.org/tex/ieeetran/
%% http://www.ctan.org/tex-archive/macros/latex/contrib/IEEEtran/
%% and
%% http://www.ieee.org/

% ================= IF YOU HAVE QUESTIONS =======================
% For questions regarding the IEEE styles, IEEE policies and
% procedures, conferences etc. please go to the IEEE web page at
% http://www.ieee.org/publications_standards/publications/authors
%
% Technical questions to bbob@lri.fr
% ===============================================================


%%*************************************************************************
%% Legal Notice:
%% This code is offered as-is without any warranty either expressed or
%% implied; without even the implied warranty of MERCHANTABILITY or
%% FITNESS FOR A PARTICULAR PURPOSE! 
%% User assumes all risk.
%% In no event shall IEEE or any contributor to this code be liable for
%% any damages or losses, including, but not limited to, incidental,
%% consequential, or any other damages, resulting from the use or misuse
%% of any information contained here.
%%
%% All comments are the opinions of their respective authors and are not
%% necessarily endorsed by the IEEE.
%%
%% This work is distributed under the LaTeX Project Public License (LPPL)
%% ( http://www.latex-project.org/ ) version 1.3, and may be freely used,
%% distributed and modified. A copy of the LPPL, version 1.3, is included
%% in the base LaTeX documentation of all distributions of LaTeX released
%% 2003/12/01 or later.
%% Retain all contribution notices and credits.
%% ** Modified files should be clearly indicated as such, including  **
%% ** renaming them and changing author support contact information. **
%%
%% File list of work: IEEEtran.cls, IEEEtran_HOWTO.pdf, bare_adv.tex,
%%                    bare_conf.tex, bare_jrnl.tex, bare_conf_compsoc.tex,
%%                    bare_jrnl_compsoc.tex, bare_jrnl_transmag.tex
%%*************************************************************************


% *** Authors should verify (and, if needed, correct) their LaTeX system  ***
% *** with the testflow diagnostic prior to trusting their LaTeX platform ***
% *** with production work. IEEE's font choices and paper sizes can       ***
% *** trigger bugs that do not appear when using other class files.       ***                          ***
% The testflow support page is at:
% http://www.michaelshell.org/tex/testflow/



\documentclass[conference]{IEEEtran}
% Some Computer Society conferences also require the compsoc mode option,
% but others use the standard conference format.
%
% If IEEEtran.cls has not been installed into the LaTeX system files,
% manually specify the path to it like:
% \documentclass[conference]{../sty/IEEEtran}

\usepackage{graphicx}
\usepackage{tabularx}
\usepackage[dvipsnames]{xcolor}
\usepackage{float}
\usepackage{rotating}
\usepackage{xstring} % for string operations
\usepackage{wasysym} % Table legend with symbols input from post-processing
\usepackage{savesym} % needed to rename \iint and \iint definitions in order to load them from MnSymbol library rather than from amsmath
\savesymbol{iint}
\savesymbol{iiint}
\usepackage{MnSymbol} % Table legend with symbols input from post-processing


% Some very useful LaTeX packages include:
% (uncomment the ones you want to load)


% *** MISC UTILITY PACKAGES ***
%
%\usepackage{ifpdf}
% Heiko Oberdiek's ifpdf.sty is very useful if you need conditional
% compilation based on whether the output is pdf or dvi.
% usage:
% \ifpdf
%   % pdf code
% \else
%   % dvi code
% \fi
% The latest version of ifpdf.sty can be obtained from:
% http://www.ctan.org/tex-archive/macros/latex/contrib/oberdiek/
% Also, note that IEEEtran.cls V1.7 and later provides a builtin
% \ifCLASSINFOpdf conditional that works the same way.
% When switching from latex to pdflatex and vice-versa, the compiler may
% have to be run twice to clear warning/error messages.






% *** CITATION PACKAGES ***
%
%\usepackage{cite}
% cite.sty was written by Donald Arseneau
% V1.6 and later of IEEEtran pre-defines the format of the cite.sty package
% \cite{} output to follow that of IEEE. Loading the cite package will
% result in citation numbers being automatically sorted and properly
% "compressed/ranged". e.g., [1], [9], [2], [7], [5], [6] without using
% cite.sty will become [1], [2], [5]--[7], [9] using cite.sty. cite.sty's
% \cite will automatically add leading space, if needed. Use cite.sty's
% noadjust option (cite.sty V3.8 and later) if you want to turn this off
% such as if a citation ever needs to be enclosed in parenthesis.
% cite.sty is already installed on most LaTeX systems. Be sure and use
% version 5.0 (2009-03-20) and later if using hyperref.sty.
% The latest version can be obtained at:
% http://www.ctan.org/tex-archive/macros/latex/contrib/cite/
% The documentation is contained in the cite.sty file itself.






% *** GRAPHICS RELATED PACKAGES ***
%
\ifCLASSINFOpdf
  % \usepackage[pdftex]{graphicx}
  % declare the path(s) where your graphic files are
  % \graphicspath{{../pdf/}{../jpeg/}}
  % and their extensions so you won't have to specify these with
  % every instance of \includegraphics
  % \DeclareGraphicsExtensions{.pdf,.jpeg,.png}
\else
  % or other class option (dvipsone, dvipdf, if not using dvips). graphicx
  % will default to the driver specified in the system graphics.cfg if no
  % driver is specified.
  % \usepackage[dvips]{graphicx}
  % declare the path(s) where your graphic files are
  % \graphicspath{{../eps/}}
  % and their extensions so you won't have to specify these with
  % every instance of \includegraphics
  % \DeclareGraphicsExtensions{.eps}
\fi
% graphicx was written by David Carlisle and Sebastian Rahtz. It is
% required if you want graphics, photos, etc. graphicx.sty is already
% installed on most LaTeX systems. The latest version and documentation
% can be obtained at: 
% http://www.ctan.org/tex-archive/macros/latex/required/graphics/
% Another good source of documentation is "Using Imported Graphics in
% LaTeX2e" by Keith Reckdahl which can be found at:
% http://www.ctan.org/tex-archive/info/epslatex/
%
% latex, and pdflatex in dvi mode, support graphics in encapsulated
% postscript (.eps) format. pdflatex in pdf mode supports graphics
% in .pdf, .jpeg, .png and .mps (metapost) formats. Users should ensure
% that all non-photo figures use a vector format (.eps, .pdf, .mps) and
% not a bitmapped formats (.jpeg, .png). IEEE frowns on bitmapped formats
% which can result in "jaggedy"/blurry rendering of lines and letters as
% well as large increases in file sizes.
%
% You can find documentation about the pdfTeX application at:
% http://www.tug.org/applications/pdftex





% *** MATH PACKAGES ***
%
%\usepackage[cmex10]{amsmath}
% A popular package from the American Mathematical Society that provides
% many useful and powerful commands for dealing with mathematics. If using
% it, be sure to load this package with the cmex10 option to ensure that
% only type 1 fonts will utilized at all point sizes. Without this option,
% it is possible that some math symbols, particularly those within
% footnotes, will be rendered in bitmap form which will result in a
% document that can not be IEEE Xplore compliant!
%
% Also, note that the amsmath package sets \interdisplaylinepenalty to 10000
% thus preventing page breaks from occurring within multiline equations. Use:
%\interdisplaylinepenalty=2500
% after loading amsmath to restore such page breaks as IEEEtran.cls normally
% does. amsmath.sty is already installed on most LaTeX systems. The latest
% version and documentation can be obtained at:
% http://www.ctan.org/tex-archive/macros/latex/required/amslatex/math/





% *** SPECIALIZED LIST PACKAGES ***
%
%\usepackage{algorithmic}
% algorithmic.sty was written by Peter Williams and Rogerio Brito.
% This package provides an algorithmic environment fo describing algorithms.
% You can use the algorithmic environment in-text or within a figure
% environment to provide for a floating algorithm. Do NOT use the algorithm
% floating environment provided by algorithm.sty (by the same authors) or
% algorithm2e.sty (by Christophe Fiorio) as IEEE does not use dedicated
% algorithm float types and packages that provide these will not provide
% correct IEEE style captions. The latest version and documentation of
% algorithmic.sty can be obtained at:
% http://www.ctan.org/tex-archive/macros/latex/contrib/algorithms/
% There is also a support site at:
% http://algorithms.berlios.de/index.html
% Also of interest may be the (relatively newer and more customizable)
% algorithmicx.sty package by Szasz Janos:
% http://www.ctan.org/tex-archive/macros/latex/contrib/algorithmicx/




% *** ALIGNMENT PACKAGES ***
%
%\usepackage{array}
% Frank Mittelbach's and David Carlisle's array.sty patches and improves
% the standard LaTeX2e array and tabular environments to provide better
% appearance and additional user controls. As the default LaTeX2e table
% generation code is lacking to the point of almost being broken with
% respect to the quality of the end results, all users are strongly
% advised to use an enhanced (at the very least that provided by array.sty)
% set of table tools. array.sty is already installed on most systems. The
% latest version and documentation can be obtained at:
% http://www.ctan.org/tex-archive/macros/latex/required/tools/


% IEEEtran contains the IEEEeqnarray family of commands that can be used to
% generate multiline equations as well as matrices, tables, etc., of high
% quality.




% *** SUBFIGURE PACKAGES ***
%\ifCLASSOPTIONcompsoc
%  \usepackage[caption=false,font=normalsize,labelfont=sf,textfont=sf]{subfig}
%\else
%  \usepackage[caption=false,font=footnotesize]{subfig}
%\fi
% subfig.sty, written by Steven Douglas Cochran, is the modern replacement
% for subfigure.sty, the latter of which is no longer maintained and is
% incompatible with some LaTeX packages including fixltx2e. However,
% subfig.sty requires and automatically loads Axel Sommerfeldt's caption.sty
% which will override IEEEtran.cls' handling of captions and this will result
% in non-IEEE style figure/table captions. To prevent this problem, be sure
% and invoke subfig.sty's "caption=false" package option (available since
% subfig.sty version 1.3, 2005/06/28) as this is will preserve IEEEtran.cls
% handling of captions.
% Note that the Computer Society format requires a larger sans serif font
% than the serif footnote size font used in traditional IEEE formatting
% and thus the need to invoke different subfig.sty package options depending
% on whether compsoc mode has been enabled.
%
% The latest version and documentation of subfig.sty can be obtained at:
% http://www.ctan.org/tex-archive/macros/latex/contrib/subfig/




% *** FLOAT PACKAGES ***
%
%\usepackage{fixltx2e}
% fixltx2e, the successor to the earlier fix2col.sty, was written by
% Frank Mittelbach and David Carlisle. This package corrects a few problems
% in the LaTeX2e kernel, the most notable of which is that in current
% LaTeX2e releases, the ordering of single and double column floats is not
% guaranteed to be preserved. Thus, an unpatched LaTeX2e can allow a
% single column figure to be placed prior to an earlier double column
% figure. The latest version and documentation can be found at:
% http://www.ctan.org/tex-archive/macros/latex/base/


%\usepackage{stfloats}
% stfloats.sty was written by Sigitas Tolusis. This package gives LaTeX2e
% the ability to do double column floats at the bottom of the page as well
% as the top. (e.g., "\begin{figure*}[!b]" is not normally possible in
% LaTeX2e). It also provides a command:
%\fnbelowfloat
% to enable the placement of footnotes below bottom floats (the standard
% LaTeX2e kernel puts them above bottom floats). This is an invasive package
% which rewrites many portions of the LaTeX2e float routines. It may not work
% with other packages that modify the LaTeX2e float routines. The latest
% version and documentation can be obtained at:
% http://www.ctan.org/tex-archive/macros/latex/contrib/sttools/
% Do not use the stfloats baselinefloat ability as IEEE does not allow
% \baselineskip to stretch. Authors submitting work to the IEEE should note
% that IEEE rarely uses double column equations and that authors should try
% to avoid such use. Do not be tempted to use the cuted.sty or midfloat.sty
% packages (also by Sigitas Tolusis) as IEEE does not format its papers in
% such ways.
% Do not attempt to use stfloats with fixltx2e as they are incompatible.
% Instead, use Morten Hogholm'a dblfloatfix which combines the features
% of both fixltx2e and stfloats:
%
% \usepackage{dblfloatfix}
% The latest version can be found at:
% http://www.ctan.org/tex-archive/macros/latex/contrib/dblfloatfix/




% *** PDF, URL AND HYPERLINK PACKAGES ***
%
%\usepackage{url}
% url.sty was written by Donald Arseneau. It provides better support for
% handling and breaking URLs. url.sty is already installed on most LaTeX
% systems. The latest version and documentation can be obtained at:
% http://www.ctan.org/tex-archive/macros/latex/contrib/url/
% Basically, \url{my_url_here}.




% *** Do not adjust lengths that control margins, column widths, etc. ***
% *** Do not use packages that alter fonts (such as pslatex).         ***
% There should be no need to do such things with IEEEtran.cls V1.6 and later.
% (Unless specifically asked to do so by the journal or conference you plan
% to submit to, of course. )


%%%%%%%%%%%%%%%%%%%%%%%%%%%%%%%%%%%%%%%%%%%%%%%%%%%%%%
% Definitions

% specify acronyms for algorithm1 (1st arg. of post-processing) and algorithm2 (2nd arg.) 
%\newcommand{\algorithmA}{algorithmB}  % first argument in the post-processing
%\newcommand{\algorithmB}{algorithmB}  % second argument in the post-processing
% for the short acronyms in the tables, adjust the following to lines if required.
%\newcommand{\algorithmAshort}{algA}  % first argument in the post-processing
%\newcommand{\algorithmBshort}{algB}  % second argument in the post-processing

% location of pictures files
\newcommand{\bbobdatapath}{ppdata/}
\input{\bbobdatapath bbob_pproc_commands.tex}
\graphicspath{{\bbobdatapath}}
%
%%%%%%%%%%%%%%%%%%%%%%%%%%%%%%%%%%%%%%%%%%%%%%%%%%%%%%%
%% pre-defined commands
\newcommand{\DIM}{\ensuremath{\mathrm{DIM}}}
\newcommand{\ERT}{\ensuremath{\mathrm{ERT}}}
\newcommand{\FEvals}{\ensuremath{\mathrm{FEvals}}}
\newcommand{\nruns}{\ensuremath{\mathrm{Nruns}}}
\newcommand{\Dfb}{\ensuremath{\Delta f_{\mathrm{best}}}}
\newcommand{\Df}{\ensuremath{\Delta f}}
\newcommand{\nbFEs}{\ensuremath{\mathrm{\#FEs}}}
\newcommand{\fopt}{\ensuremath{f_\mathrm{opt}}}
\newcommand{\ftarget}{\ensuremath{f_\mathrm{t}}}
\newcommand{\CrE}{\ensuremath{\mathrm{CrE}}}
\newcommand{\change}[1]{{\color{red} #1}}

    \renewcommand{\topfraction}{1}	% max fraction of floats at top
    \renewcommand{\bottomfraction}{1} % max fraction of floats at bottom
    %   Parameters for TEXT pages (not float pages):
    \setcounter{topnumber}{3}
    \setcounter{bottomnumber}{3}
    \setcounter{totalnumber}{3}     % 2 may work better
    \setcounter{dbltopnumber}{4}    % for 2-column pages
    \renewcommand{\dbltopfraction}{1}	% fit big float above 2-col. text
    \renewcommand{\textfraction}{0.0}	% allow minimal text w. figs
    %   Parameters for FLOAT pages (not text pages):
    \renewcommand{\floatpagefraction}{0.80}	% require fuller float pages
    % N.B.: floatpagefraction MUST be less than topfraction !!
    \renewcommand{\dblfloatpagefraction}{0.7}	% require fuller float pages

%%%%%%%%%%%%%%%%%%%%%%%%%%%%%%%%%%%%%%%%%%%%%%%%%%%%%%


% correct bad hyphenation here
\hyphenation{op-tical net-works semi-conduc-tor}


\begin{document}
%
% paper title
% Titles are generally capitalized except for words such as a, an, and, as,
% at, but, by, for, in, nor, of, on, or, the, to and up, which are usually
% not capitalized unless they are the first or last word of the title.
% Linebreaks \\ can be used within to get better formatting as desired.
% Do not put math or special symbols in the title.
\title{Black-Box Optimization Benchmarking Template for the Comparison of Two Algorithms on the Noisy Testbed\\ {\large Draft version}}

% author names and affiliations
% use a multiple column layout for up to three different
% affiliations
\author{\IEEEauthorblockN{Firstname Lastname}
\IEEEauthorblockA{Affiliation\\
Address\\
Email: youremail@yourinstitution.org}\thanks{Submission deadline: December 19, 2014.}
%\and
%\IEEEauthorblockN{Homer Simpson}
%\IEEEauthorblockA{Twentieth Century Fox\\
%Springfield, USA\\
%Email: homer@thesimpsons.com}
%\and
%\IEEEauthorblockN{James Kirk\\ and Montgomery Scott}
%\IEEEauthorblockA{Starfleet Academy\\
%San Francisco, California 96678-2391\\
%Telephone: (800) 555--1212\\
%Fax: (888) 555--1212}
}

% conference papers do not typically use \thanks and this command
% is locked out in conference mode. If really needed, such as for
% the acknowledgment of grants, issue a \IEEEoverridecommandlockouts
% after \documentclass
\IEEEoverridecommandlockouts

% for over three affiliations, or if they all won't fit within the width
% of the page, use this alternative format:
% 
%\author{\IEEEauthorblockN{Michael Shell\IEEEauthorrefmark{1},
%Homer Simpson\IEEEauthorrefmark{2},
%James Kirk\IEEEauthorrefmark{3}, 
%Montgomery Scott\IEEEauthorrefmark{3} and
%Eldon Tyrell\IEEEauthorrefmark{4}}
%\IEEEauthorblockA{\IEEEauthorrefmark{1}School of Electrical and Computer Engineering\\
%Georgia Institute of Technology,
%Atlanta, Georgia 30332--0250\\ Email: see http://www.michaelshell.org/contact.html}
%\IEEEauthorblockA{\IEEEauthorrefmark{2}Twentieth Century Fox, Springfield, USA\\
%Email: homer@thesimpsons.com}
%\IEEEauthorblockA{\IEEEauthorrefmark{3}Starfleet Academy, San Francisco, California 96678-2391\\
%Telephone: (800) 555--1212, Fax: (888) 555--1212}
%\IEEEauthorblockA{\IEEEauthorrefmark{4}Tyrell Inc., 123 Replicant Street, Los Angeles, California 90210--4321}}




% use for special paper notices
%\IEEEspecialpapernotice{(Invited Paper)}




% make the title area
\maketitle


\begin{abstract}
%\boldmath
The abstract goes here.
\end{abstract}
% IEEEtran.cls defaults to using nonbold math in the Abstract.
% This preserves the distinction between vectors and scalars. However,
% if the conference you are submitting to favors bold math in the abstract,
% then you can use LaTeX's standard command \boldmath at the very start
% of the abstract to achieve this. Many IEEE journals/conferences frown on
% math in the abstract anyway.

% no keywords




% For peer review papers, you can put extra information on the cover
% page as needed:
% \ifCLASSOPTIONpeerreview
% \begin{center} \bfseries EDICS Category: 3-BBND \end{center}
% \fi
%
% For peerreview papers, this IEEEtran command inserts a page break and
% creates the second title. It will be ignored for other modes.
\IEEEpeerreviewmaketitle


% \section{Introduction}
%
% \section{Algorithm Presentation}
%
% \section{Experimental Procedure}
%
%%%%%%%%%%%%%%%%%%%%%%%%%%%%%%%%%%%%%%%%%%%%%%%%%%%%%%%%%%%%%%%%%%%%%%%%%%%%%%%
\section{CPU Timing}
%%%%%%%%%%%%%%%%%%%%%%%%%%%%%%%%%%%%%%%%%%%%%%%%%%%%%%%%%%%%%%%%%%%%%%%%%%%%%%%
% note that the following text is just a proposal and can/should be changed to your needs:
In order to evaluate the CPU timing of the algorithm, we have run the \change{MY-ALGORITHM-NAME} on the function $f_{8}$ with restarts for at least 30 seconds and until a maximum budget equal to \change{$400 (D + 2)$} is reached. The code was run on a \change{Mac Intel(R) Core(TM) i5-2400S CPU @ 2.50GHz} with \change{1} processor and \change{4} cores. The time per function evaluation for dimensions 2, 3, 5, 10, 20\change{, 40} equals \change{$x.x$}, \change{$x.x$}, \change{$x.x$}, \change{$xx$}, \change{$xxx$}\change{, and $xxx$} milliseconds respectively. 

\change{repeat the above for the second algorithm}

%%%%%%%%%%%%%%%%%%%%%%%%%%%%%%%%%%%%%%%%%%%%%%%%%%%%%%%%%%%%%%%%%%%%%%%%%%%%%%%
\section{Results}
%%%%%%%%%%%%%%%%%%%%%%%%%%%%%%%%%%%%%%%%%%%%%%%%%%%%%%%%%%%%%%%%%%%%%%%%%%%%%%%

Results from experiments according to \cite{hansen2012exp} on the benchmark
functions given in \cite{wp200902_2010,hansen2012noi} are presented in
Figures~\ref{fig:scaling}, \ref{fig:scatterplots} and \ref{fig:RLDs} and
in Tables~\ref{tab:ERTs1to15} and~\ref{tab:ERTs16to30}. The \textbf{expected running time (ERT)}, used in the figures and table, depends on a
given target function value, $\ftarget=\fopt+\Delta\ftarget$, and is computed over all relevant trials
as the number of function evaluations executed during each trial while the best
function value did not reach \ftarget, summed over all trials
and divided by the number of trials that actually reached \ftarget\
\cite{hansen2012exp,price1997dev}. 
\textbf{Statistical significance} is tested with the rank-sum test for a given
target $\Delta\ftarget$ ($10^{-8}$ as in Figure~\ref{fig:scaling}) using,
for each trial, either the number of needed function evaluations to reach
$\Delta\ftarget$ (inverted and multiplied by $-1$), or, if the target was not
reached, the best $\Df$-value achieved, measured only up to the smallest number
of overall function evaluations for any unsuccessful trial under consideration. 

%%%%%%%%%%%%%%%%%%%%%%%%%%%%%%%%%%%%%%%%%%%%%%%%%%%%%%%%%%%%%%%%%%%%%%%%%%%%%%%
%%%%%%%%%%%%%%%%%%%%%%%%%%%%%%%%%%%%%%%%%%%%%%%%%%%%%%%%%%%%%%%%%%%%%%%%%%%%%%%

% Scaling of ERT with dimension

%%%%%%%%%%%%%%%%%%%%%%%%%%%%%%%%%%%%%%%%%%%%%%%%%%%%%%%%%%%%%%%%%%%%%%%%%%%%%%%
\begin{figure*}
\centering
\begin{tabular}{@{}c@{}c@{}c@{}c@{}c@{}}
\includegraphics[width=0.204\textwidth, trim= 0.7cm 0.8cm 0.5cm 0.5cm, clip]{ppfigs_f101}&
\includegraphics[width=0.193\textwidth, trim= 1.8cm 0.8cm 0.5cm 0.5cm, clip]{ppfigs_f104}&
\includegraphics[width=0.193\textwidth, trim= 1.8cm 0.8cm 0.5cm 0.5cm, clip]{ppfigs_f107}&
\includegraphics[width=0.193\textwidth, trim= 1.8cm 0.8cm 0.5cm 0.5cm, clip]{ppfigs_f110}&
\includegraphics[width=0.193\textwidth, trim= 1.8cm 0.8cm 0.5cm 0.5cm, clip]{ppfigs_f113}\\
\includegraphics[width=0.204\textwidth, trim= 0.7cm 0.8cm 0.5cm 0.5cm, clip]{ppfigs_f102}&
\includegraphics[width=0.193\textwidth, trim= 1.8cm 0.8cm 0.5cm 0.5cm, clip]{ppfigs_f105}&
\includegraphics[width=0.204\textwidth, trim= 1.8cm 0.8cm 0.5cm 0.5cm, clip]{ppfigs_f108}&
\includegraphics[width=0.193\textwidth, trim= 1.8cm 0.8cm 0.5cm 0.5cm, clip]{ppfigs_f111}&
\includegraphics[width=0.193\textwidth, trim= 1.8cm 0.8cm 0.5cm 0.5cm, clip]{ppfigs_f114}\\
\includegraphics[width=0.204\textwidth, trim= 0.7cm 0.8cm 0.5cm 0.5cm, clip]{ppfigs_f103}&
\includegraphics[width=0.193\textwidth, trim= 1.8cm 0.8cm 0.5cm 0.5cm, clip]{ppfigs_f106}&
\includegraphics[width=0.193\textwidth, trim= 1.8cm 0.8cm 0.5cm 0.5cm, clip]{ppfigs_f109}&
\includegraphics[width=0.193\textwidth, trim= 1.8cm 0.8cm 0.5cm 0.5cm, clip]{ppfigs_f112}&
\includegraphics[width=0.193\textwidth, trim= 1.8cm 0.8cm 0.5cm 0.5cm, clip]{ppfigs_f115}\\\hline
\includegraphics[width=0.204\textwidth, trim= 0.7cm 0.8cm 0.5cm 0.5cm, clip]{ppfigs_f116}&
\includegraphics[width=0.193\textwidth, trim= 1.8cm 0.8cm 0.5cm 0.5cm, clip]{ppfigs_f119}&
\includegraphics[width=0.193\textwidth, trim= 1.8cm 0.8cm 0.5cm 0.5cm, clip]{ppfigs_f122}&
\includegraphics[width=0.193\textwidth, trim= 1.8cm 0.8cm 0.5cm 0.5cm, clip]{ppfigs_f125}&
\includegraphics[width=0.193\textwidth, trim= 1.8cm 0.8cm 0.5cm 0.5cm, clip]{ppfigs_f128}\\
\includegraphics[width=0.204\textwidth, trim= 0.7cm 0.8cm 0.5cm 0.5cm, clip]{ppfigs_f117}&
\includegraphics[width=0.193\textwidth, trim= 1.8cm 0.8cm 0.5cm 0.5cm, clip]{ppfigs_f120}&
\includegraphics[width=0.193\textwidth, trim= 1.8cm 0.8cm 0.5cm 0.5cm, clip]{ppfigs_f123}&
\includegraphics[width=0.193\textwidth, trim= 1.8cm 0.8cm 0.5cm 0.5cm, clip]{ppfigs_f126}&
\includegraphics[width=0.193\textwidth, trim= 1.8cm 0.8cm 0.5cm 0.5cm, clip]{ppfigs_f129}\\
\includegraphics[width=0.204\textwidth, trim= 0.7cm 0.0cm 0.5cm 0.5cm, clip]{ppfigs_f118}&
\includegraphics[width=0.193\textwidth, trim= 1.8cm 0.0cm 0.5cm 0.5cm, clip]{ppfigs_f121}&
\includegraphics[width=0.193\textwidth, trim= 1.8cm 0.0cm 0.5cm 0.5cm, clip]{ppfigs_f124}&
\includegraphics[width=0.193\textwidth, trim= 1.8cm 0.0cm 0.5cm 0.5cm, clip]{ppfigs_f127}&
\includegraphics[width=0.193\textwidth, trim= 1.8cm 0.0cm 0.5cm 0.5cm, clip]{ppfigs_f130}
\end{tabular}
\vspace*{-0.2cm}
\caption[Expected running time (\ERT) divided by dimension
versus dimension in log-log presentation]{\label{fig:scaling}
\bbobppfigslegend{$f_{101}$ and $f_{130}$}. % \algorithmA and \algorithmB can be (re)-defined above
}
\end{figure*}

%%%%%%%%%%%%%%%%%%%%%%%%%%%%%%%%%%%%%%%%%%%%%%%%%%%%%%%%%%%%%%%%%%%%%%%%%%%%%%%
%%%%%%%%%%%%%%%%%%%%%%%%%%%%%%%%%%%%%%%%%%%%%%%%%%%%%%%%%%%%%%%%%%%%%%%%%%%%%%%
 
% Scatter plots per function.

%%%%%%%%%%%%%%%%%%%%%%%%%%%%%%%%%%%%%%%%%%%%%%%%%%%%%%%%%%%%%%%%%%%%%%%%%%%%%%%
\begin{figure*}
\small
\centering
\begin{tabular}{@{}c@{}*{5}{@{}c@{}}}
 & {\sf\footnotesize 101 Sphere (moderate)} & {\sf\footnotesize 104 Rosenbrock (moderate)} & {\sf\footnotesize 107 Sphere} & {\sf\footnotesize 110 Rosenbrock} & {\sf\footnotesize 113 Step-ellipsoid} \\
\begin{turn}{90}\parbox{0.175\textwidth}{\centering\sf Gauss noise}\end{turn} &
    \includegraphics[height=0.175\textwidth, trim= 36mm 9mm 20mm 9mm, clip]{ppscatter_f101}&
    \includegraphics[height=0.175\textwidth, trim= 36mm 9mm 20mm 9mm, clip]{ppscatter_f104}&
    \includegraphics[height=0.175\textwidth, trim= 36mm 9mm 20mm 9mm, clip]{ppscatter_f107}&
    \includegraphics[height=0.175\textwidth, trim= 36mm 9mm 20mm 9mm, clip]{ppscatter_f110}&
    \includegraphics[height=0.175\textwidth, trim= 36mm 9mm 20mm 9mm, clip]{ppscatter_f113}\\
 & {\sf\footnotesize 102 Sphere (moderate)} & {\sf\footnotesize 105 Rosenbrock (moderate)} & {\sf\footnotesize 108 Sphere} & {\sf\footnotesize 111 Rosenbrock} & {\sf\footnotesize 114 Step-ellipsoid}\\
\begin{turn}{90}\parbox{0.175\textwidth}{\centering\sf uniform noise}\end{turn} &
    \includegraphics[height=0.175\textwidth, trim= 36mm 9mm 20mm 9mm, clip]{ppscatter_f102}&
    \includegraphics[height=0.175\textwidth, trim= 36mm 9mm 20mm 9mm, clip]{ppscatter_f105}&
    \includegraphics[height=0.175\textwidth, trim= 36mm 9mm 20mm 9mm, clip]{ppscatter_f108}&
    \includegraphics[height=0.175\textwidth, trim= 36mm 9mm 20mm 9mm, clip]{ppscatter_f111}&
    \includegraphics[height=0.175\textwidth, trim= 36mm 9mm 20mm 9mm, clip]{ppscatter_f114}\\
 & {\sf\footnotesize 103 Sphere (moderate)} & {\sf\footnotesize 106 Rosenbrock (moderate)} & {\sf\footnotesize 109 Sphere} & {\sf\footnotesize 112 Rosenbrock} & {\sf\footnotesize 115 Step-ellipsoid}\\
\begin{turn}{90}\parbox{0.175\textwidth}{\centering\sf Cauchy noise}\end{turn} &
    \includegraphics[height=0.175\textwidth, trim= 36mm 9mm 20mm 9mm, clip]{ppscatter_f103}&
    \includegraphics[height=0.175\textwidth, trim= 36mm 9mm 20mm 9mm, clip]{ppscatter_f106}&
    \includegraphics[height=0.175\textwidth, trim= 36mm 9mm 20mm 9mm, clip]{ppscatter_f109}&
    \includegraphics[height=0.175\textwidth, trim= 36mm 9mm 20mm 9mm, clip]{ppscatter_f112}&
    \includegraphics[height=0.175\textwidth, trim= 36mm 9mm 20mm 9mm, clip]{ppscatter_f115}\\\hline
 & {\sf\footnotesize 116 Ellipsoid} & {\sf\footnotesize 119 Sum of diff.\ powers} & {\sf\footnotesize 122 Schaffer F7} & {\sf\footnotesize 125 Griewank-Rosenbrock} & {\sf\footnotesize 128 Gallagher}\\
\begin{turn}{90}\parbox{0.175\textwidth}{\centering\sf Gauss noise}\end{turn} &
    \includegraphics[height=0.175\textwidth, trim= 36mm 9mm 20mm 9mm, clip]{ppscatter_f116}&
    \includegraphics[height=0.175\textwidth, trim= 36mm 9mm 20mm 9mm, clip]{ppscatter_f119}&
    \includegraphics[height=0.175\textwidth, trim= 36mm 9mm 20mm 9mm, clip]{ppscatter_f122}&
    \includegraphics[height=0.175\textwidth, trim= 36mm 9mm 20mm 9mm, clip]{ppscatter_f125}&
    \includegraphics[height=0.175\textwidth, trim= 36mm 9mm 20mm 9mm, clip]{ppscatter_f128}\\
 & {\sf\footnotesize 117 Ellipsoid} & {\sf\footnotesize 120 Sum of diff.\ powers} & {\sf\footnotesize 123 Schaffer F7} & {\sf\footnotesize 126 Griewank-Rosenbrock} & {\sf\footnotesize 129 Gallagher}\\
\begin{turn}{90}\parbox{0.175\textwidth}{\centering\sf uniform noise}\end{turn} &
    \includegraphics[height=0.175\textwidth, trim= 36mm 9mm 20mm 9mm, clip]{ppscatter_f117}&
    \includegraphics[height=0.175\textwidth, trim= 36mm 9mm 20mm 9mm, clip]{ppscatter_f120}&
    \includegraphics[height=0.175\textwidth, trim= 36mm 9mm 20mm 9mm, clip]{ppscatter_f123}&
    \includegraphics[height=0.175\textwidth, trim= 36mm 9mm 20mm 9mm, clip]{ppscatter_f126}&
    \includegraphics[height=0.175\textwidth, trim= 36mm 9mm 20mm 9mm, clip]{ppscatter_f129}\\
& {\sf\footnotesize 118 Ellipsoid} & {\sf\footnotesize 121 Sum of diff.\ powers} & {\sf\footnotesize 124 Schaffer F7} & {\sf\footnotesize 127 Griewank-Rosenbrock} & {\sf\footnotesize 130 Gallagher}\\
\begin{turn}{90}\parbox{0.175\textwidth}{~\centering\sf Cauchy noise}\end{turn} &
    \includegraphics[height=0.175\textwidth, trim= 36mm 9mm 20mm 9mm, clip]{ppscatter_f118}&
    \includegraphics[height=0.175\textwidth, trim= 36mm 9mm 20mm 9mm, clip]{ppscatter_f121}&
    \includegraphics[height=0.175\textwidth, trim= 36mm 9mm 20mm 9mm, clip]{ppscatter_f124}&
    \includegraphics[height=0.175\textwidth, trim= 36mm 9mm 20mm 9mm, clip]{ppscatter_f127}&
    \includegraphics[height=0.175\textwidth, trim= 36mm 9mm 20mm 9mm, clip]{ppscatter_f130}
\end{tabular}
\caption{\label{fig:scatterplots}
\bbobppscatterlegend{$f_{101}$--$f_{130}$} 
}
\end{figure*}


%%%%%%%%%%%%%%%%%%%%%%%%%%%%%%%%%%%%%%%%%%%%%%%%%%%%%%%%%%%%%%%%%%%%%%%%%%%%%%%
%%%%%%%%%%%%%%%%%%%%%%%%%%%%%%%%%%%%%%%%%%%%%%%%%%%%%%%%%%%%%%%%%%%%%%%%%%%%%%%
\newcommand{\rot}[2][2.5]{
  \hspace*{-3.5\baselineskip}%
  \begin{rotate}{90}\hspace{#1em}#2
  \end{rotate}}
%%%%%%%%%%%%%%%%%%%%%%%%%%%%%%%%%%%%%%%%%%%%%%%%%%%%%%%%%%%%%%%%%%%%%%%%%%%%%%%
%%%%%%%%%%%%%%%%%%%%%%%%%%%%%%%%%%%%%%%%%%%%%%%%%%%%%%%%%%%%%%%%%%%%%%%%%%%%%%%


%%%%%%%%%%%%%%%%%%%%%%%%%%%%%%%%%%%%%%%%%%%%%%%%%%%%%%%%%%%%%%%%%%%%%%%%%%%%%%%
%%%%%%%%%%%%%%%%%%%%%%%%%%%%%%%%%%%%%%%%%%%%%%%%%%%%%%%%%%%%%%%%%%%%%%%%%%%%%%%
 
% Empirical cumulative distribution functions (ECDFs) per function group.

%%%%%%%%%%%%%%%%%%%%%%%%%%%%%%%%%%%%%%%%%%%%%%%%%%%%%%%%%%%%%%%%%%%%%%%%%%%%%%%
\begin{figure*}
 \begin{tabular}{l@{\hspace*{-0.025\textwidth}}l|l@{\hspace*{-0.025\textwidth}}l}
 \multicolumn{2}{c}{5-D} & \multicolumn{2}{c}{20-D} \\
 \rot{moderate noise}
 \includegraphics[width=0.268\textwidth,trim=     0mm 0mm 0mm 15mm, clip]{pprldistr_05D_nzmod} & 
 \includegraphics[width=0.2375\textwidth,trim=2.3cm 0mm 0mm 15mm, clip]{pplogabs_05D_nzmod} &
 \includegraphics[width=0.268\textwidth,trim=     0mm 0mm 0mm 15mm, clip]{pprldistr_20D_nzmod} &
 \includegraphics[width=0.2375\textwidth,trim=2.3cm 0mm 0mm 15mm, clip]{pplogabs_20D_nzmod} \\
 \rot{severe noise}
 \includegraphics[width=0.268\textwidth,trim=     0mm 0mm 0mm 15mm, clip]{pprldistr_05D_nzsev} & 
 \includegraphics[width=0.2375\textwidth,trim=2.3cm 0mm 0mm 15mm, clip]{pplogabs_05D_nzsev} &
 \includegraphics[width=0.268\textwidth,trim=     0mm 0mm 0mm 15mm, clip]{pprldistr_20D_nzsev} & 
 \includegraphics[width=0.2375\textwidth,trim=2.3cm 0mm 0mm 15mm, clip]{pplogabs_20D_nzsev}\\
 \rot[0.5]{severe noise multimod.}
 \includegraphics[width=0.268\textwidth,trim=     0mm 0mm 0mm 15mm, clip]{pprldistr_05D_nzsmm} & 
 \includegraphics[width=0.2375\textwidth,trim=2.3cm 0mm 0mm 15mm, clip]{pplogabs_05D_nzsmm} &
 \includegraphics[width=0.268\textwidth,trim=     0mm 0mm 0mm 15mm, clip]{pprldistr_20D_nzsmm} &
 \includegraphics[width=0.2375\textwidth,trim=2.3cm 0mm 0mm 15mm, clip]{pplogabs_20D_nzsmm}\\
 \rot{all functions}
 \includegraphics[width=0.268\textwidth]{pprldistr_05D_nzall} & 
 \includegraphics[width=0.2375\textwidth,trim=2.3cm 0mm 0mm 0mm, clip]{pplogabs_05D_nzall} &
 \includegraphics[width=0.268\textwidth]{pprldistr_20D_nzall} &
 \includegraphics[width=0.2375\textwidth,trim=2.3cm 0mm 0mm 0mm, clip]{pplogabs_20D_nzall} 
 \end{tabular}
 \caption{\label{fig:RLDs}
 \bbobpprldistrlegendtwo{}
 }
\end{figure*}



%%%%%%%%%%%%%%%%%%%%%%%%%%%%%%%%%%%%%%%%%%%%%%%%%%%%%%%%%%%%%%%%%%%%%%%%%%%%%%%
%%%%%%%%%%%%%%%%%%%%%%%%%%%%%%%%%%%%%%%%%%%%%%%%%%%%%%%%%%%%%%%%%%%%%%%%%%%%%%%
 
% Table showing the expected running time (ERT in number of function
% evaluations) divided by the best ERT measured during BBOB-2009 (given in the
% first row of each cell) for functions $f_{101}$--$f_{115}$.

%%%%%%%%%%%%%%%%%%%%%%%%%%%%%%%%%%%%%%%%%%%%%%%%%%%%%%%%%%%%%%%%%%%%%%%%%%%%%%%
\begin{table*}
\centering
\hfill5-D\hfill20-D\hfill~\\[1ex]
\tiny
\mbox{%
\input{\bbobdatapath pptable2_05D_nzall0}\hspace*{1em}%
\input{\bbobdatapath pptable2_20D_nzall0}}
\vspace*{1em}
\caption{\label{tab:ERTs1to15} 
\bbobpptablestwolegend{60}
}
\end{table*}



%%%%%%%%%%%%%%%%%%%%%%%%%%%%%%%%%%%%%%%%%%%%%%%%%%%%%%%%%%%%%%%%%%%%%%%%%%%%%%%
%%%%%%%%%%%%%%%%%%%%%%%%%%%%%%%%%%%%%%%%%%%%%%%%%%%%%%%%%%%%%%%%%%%%%%%%%%%%%%%
 
% Table showing the expected running time (ERT in number of function
% evaluations) divided by the best ERT measured during BBOB-2009 (given in the
% first row of each cell) for functions $f_{116}$--$f_{130}$.

%%%%%%%%%%%%%%%%%%%%%%%%%%%%%%%%%%%%%%%%%%%%%%%%%%%%%%%%%%%%%%%%%%%%%%%%%%%%%%%
\begin{table*}
\centering
\hfill5-D\hfill20-D\hfill~\\[1ex]
\tiny
\mbox{%
\input{\bbobdatapath pptable2_05D_nzall1}\hspace*{1em}%
\input{\bbobdatapath pptable2_20D_nzall1}}
\vspace*{1em}
\caption{\label{tab:ERTs16to30} Relative \ERT\ in number of $f$-evaluations, see Table~\ref{tab:ERTs1to15} for details. 
}
\end{table*}


%%%%%%%%%%%%%%%%%%%%%%%%%%%%%%%%%%%%%%%%%%%%%%%%%%%%%%%%%%%%%%%%%%%%%%%%%%%%%%%
%%%%%%%%%%%%%%%%%%%%%%%%%%%%%%%%%%%%%%%%%%%%%%%%%%%%%%%%%%%%%%%%%%%%%%%%%%%%%%%

% use section* for acknowledgement
\section*{Acknowledgment}
The authors would like to thank...



%%%%%%%%%%%%%%%%%%%%%%%%%%%%%%%%%%%%%%%%%%%%%%%%%%%%%%%%%%%%%%%%%%%%%%%%%%%%%%%
%%%%%%%%%%%%%%%%%%%%%%%%%%%%%%%%%%%%%%%%%%%%%%%%%%%%%%%%%%%%%%%%%%%%%%%%%%%%%%%

% trigger a \newpage just before the given reference
% number - used to balance the columns on the last page
% adjust value as needed - may need to be readjusted if
% the document is modified later
%\IEEEtriggeratref{8}
% The "triggered" command can be changed if desired:
%\IEEEtriggercmd{\enlargethispage{-5in}}

% references section

% can use a bibliography generated by BibTeX as a .bbl file
% BibTeX documentation can be easily obtained at:
% http://www.ctan.org/tex-archive/biblio/bibtex/contrib/doc/
% The IEEEtran BibTeX style support page is at:
% http://www.michaelshell.org/tex/ieeetran/bibtex/
\bibliographystyle{IEEEtran}
% argument is your BibTeX string definitions and bibliography database(s)
\bibliography{bbob}  % bbob.bib is the name of the Bibliography in this case
%
% <OR> manually copy in the resultant .bbl file
% set second argument of \begin to the number of references
% (used to reserve space for the reference number labels box)
%\begin{thebibliography}{1}
%
%\bibitem{IEEEhowto:kopka}
%H.~Kopka and P.~W. Daly, \emph{A Guide to \LaTeX}, 3rd~ed.\hskip 1em plus
%  0.5em minus 0.4em\relax Harlow, England: Addison-Wesley, 1999.
%
%\end{thebibliography}


% that's all folks
\end{document}


